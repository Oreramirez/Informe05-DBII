\section{Marco Teorico} 

\begin{itemize}
\subsection{ Docker:}
	\item Tener un docker que provea el gestor de base de datos es muy útil porque se reducen tiempos de instalación y configuración y en caso de tener un error muy grave en la configuración es tan sencillo resolverlo como borrar el contenedor y crear uno nuevo.
          \item Los contenedores funcionan bien para desarrollo y tal vez algunos ambientes de evaluación para el cliente, pero para ambientes productivos para nada se recomiendan, en estos casos siempre será lo mejor que se cuente con una base de datos instalada en el servidor.
         \item Sirven para desplegar aplicaciones en un entorno virtual aislado, pero sin el overhead de tener un Sistema Operativo (SO) nuevo como se tiene en una Virtual Machine (VM).

\subsection{Oracle Database en Docker:}
	\item Los productos de Oracle son compatibles con Docker si el sistema operativo del host es Oracle Linux 7, pero no necesita usar un host OL7 para que esto funcione. Puedes ver cómo instalar Docker en OL7 .
	\item Usar imágenes de Oracle Container Registry o de Docker Store tiene la ventaja que los binarios de instalación vienen incluidos, lo que no es permitido por licencia en el resto de las distribuciones. 


\end{itemize}







